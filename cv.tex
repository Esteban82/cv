% !TEX TS-program = xelatex
\documentclass[11pt, letter]{article}
\usepackage[utf8]{inputenc}
\usepackage[vmargin=3cm, hmargin=2.5cm]{geometry}
\usepackage{enumitem} % fancier lists

% Use Palatino font
\usepackage{fontspec}
\setmainfont[BoldFont=texgyrepagella-bold.otf, ItalicFont=texgyrepagella-italic.otf, BoldItalicFont=texgyrepagella-bolditalic.otf]{texgyrepagella-regular.otf}

% Define custom environments
\usepackage{environ}

% Disable hyphenation in the CV
\usepackage[none]{hyphenat}

% Configure paragraph spacings
\setlength{\parindent}{0cm} % Remove indentations

% Configure the font style for sections
\usepackage{titlesec}
\titleformat{\section}
    [hang] % the default shape for sections
    {\normalfont\LARGE\bfseries} % format
    {} % label
    {0pt} % separation (left separation for hang)
    {} % before title
    [\titlerule] % after title
\titleformat{\subsection}
    [hang] % the default shape for sections
    {\large\MakeUppercase} % format
    {} % label
    {0pt} % separation (left separation for hang)
    {} % before title

% Disable number of sections. Use this instead of "section*" so that the
% sections still appear as PDF bookmarks. Otherwise, would have to add the
% table of contents entries manually.
\makeatletter
\renewcommand{\@seccntformat}[1]{}
\makeatother

% Configure fancyhdr
\usepackage{fancyhdr}
\usepackage{lastpage} % get the total page numbers (\pageref{LastPage})
\fancyhf{}  % clear current head and footer
\fancyfoot[C]{\small \thepage{} of~\pageref{LastPage}}
\renewcommand{\headrulewidth}{0cm} % remove header rule
\pagestyle{fancy}

% Define new colors
\usepackage{xcolor}
\definecolor{mygray}{HTML}{555555}

 % Define multipage tables (used for cv entries)
\usepackage{tabularx}
\usepackage{ltablex}

% Boolean operators
\usepackage{ifthen}

% -----------
% Information
% -----------
\newcommand{\nametitle}{Dr.}
\newcommand{\firstname}{Santiago}
\newcommand{\middlename}{Rubén}
\newcommand{\familyname}{Soler}
\newcommand{\email}{santiago.r.soler@gmail.com}
\newcommand{\website}{www.santisoler.com}
\newcommand{\github}{santisoler}
\newcommand{\orcid}{0000-0001-9202-5317}


% ------------
% Institutions
% ------------
\newcommand{\compgeolab}{%
    \href{https://www.compgeolab.com}{Computer-Oriented Geoscience Lab}%
}
\newcommand{\conicet}{%
    Consejo Nacional de Investigaciones Científicas y Técnicas%
}
\newcommand{\igsv}{Instituto Geofísico Sismológico Volponi}
\newcommand{\fcefn}{%
    Facultad de Ciencias Exactas, Físicas y Naturales%
}
\newcommand{\unsj}{Universidad Nacional de San Juan}
\newcommand{\ecen}{Escuela de Ciencias Exactas y Naturales}
\newcommand{\fceia}{%
    Facultad de Ciencias Exactas, Ingeniería y Agrimensura%
}
\newcommand{\unr}{Universidad Nacional de Rosario}
\newcommand{\gif}{Geophysical Inversion Facility}
\newcommand{\eoas}{Department of Earth, Ocean and Atmospheric Sciences}
\newcommand{\ubc}{University of British Columbia}


% -------
% Authors
% -------
\newcommand{\me}{\href{https://www.santisoler.com}{Santiago Soler}}
\newcommand{\agustina}{\href{https://aguspesce.github.io}{Agustina Pesce}}
\newcommand{\andrea}{\href{https://www.andreabalza.com/}{Andrea Balza Morales}}
\newcommand{\guido}{Guido M. Gianni}
\newcommand{\leo}{\href{https://www.leouieda.com}{Leonardo Uieda}}
\newcommand{\lindsey}{\href{https://lindseyjh.ca/}{Lindsey Heagy}}
\newcommand{\mario}{Mario Giménez}
\newcommand{\remirampin}{Rémi Rampin}
\newcommand{\hugovankemenade}{Hugo vanKemenade}
\newcommand{\matthewturk}{Matthew Turk}
\newcommand{\danshapero}{Daniel Shapero}
\newcommand{\andersonbanihirwe}{Anderson Banihirwe}
\newcommand{\johnleeman}{John Leeman}
\newcommand{\orlando}{Orlando Álvarez}
\newcommand{\folguera}{Andrés Folguera}
\newcommand{\wenjinchen}{Wenjin Chen}
\newcommand{\pichu}{Héctor P.A. García}
\newcommand{\mae}{Marianela N. Lupari}
\newcommand{\marcos}{Marcos A. Sánchez}
\newcommand{\paco}{Francisco Ruiz}
\newcommand{\fede}{Federico G. Lince Klinger}

% --------------------------
% Variables and environments
% --------------------------
% New variables
\newcommand{\fullname}{\nametitle{} \firstname{} \middlename{} \familyname}

% Titles and headings
\newcommand{\maintitle}[1]{%
    \begin{flushleft}
        \textbf{\Huge #1}
    \end{flushleft}
}
\newcommand{\subtitle}[1]{%
    \begin{flushleft}
        {\LARGE #1}
    \end{flushleft}
}
\newcommand{\affiliation}[1]{%
    \begin{flushleft}
        {\large #1}
    \end{flushleft}
}

% Entries
\newcommand{\entriespad}{0.75em}
\NewEnviron{cventries}{
    \vspace{-1em}
    \begin{tabularx}{\textwidth}{p{0.12\textwidth} p{0.82\textwidth}}
        \BODY{}
    \end{tabularx}
}
\newcommand{\singleline}[2]{{#1} & {#2} \vspace{\entriespad} \\}
\newcommand{\multiline}[3]{{#1} & {\textbf{#2} \newline {#3}} \vspace{0.6em} \\}
\newcommand{\paper}[5]{%
    {#1}
    &
    {\textbf{#2}, \emph{#3} \newline {#4} \newline doi:~\DOI{#5}}
    \vspace{\entriespad} \\
}
\newcommand{\talk}[5]{%
    {#1}
    &
    {%
        \textbf{#2}, \emph{#3}%
        \ifthenelse{\equal{#4}{}}{}{\newline {#4}}%
        \ifthenelse{\equal{#5}{}}{}{\newline doi:~\DOI{#5}}%
    }
    \vspace{\entriespad} \\
}

% Macros
\newcommand{\MAIL}[1]{%
    \href{mailto:#1}{#1}
}
\newcommand{\GITHUB}[1]{%
    \href{https://github.com/#1}{github.com/#1}
}
\newcommand{\ORCID}[1]{%
    \href{https://orcid.org/#1}{#1}
}
\newcommand{\WEBSITE}[1]{%
    \href{https://#1}{#1}
}
\newcommand{\DOI}[1]{%
    \href{https://www.doi.org/#1}{#1}
}

% ------------------
% Configure hyperref
% ------------------
\usepackage{hyperref}
\hypersetup{
    colorlinks,
    allcolors=[rgb]{0, 0.451, 0.753},
    pdftitle={\fullname{}'s Curriculum Vitae},
    pdfauthor={\fullname},
}
\urlstyle{same}

% ------------------------------------------------------------------------

\begin{document}


% ------
% Header
% ------
\maintitle{\fullname}

\begin{minipage}[t]{0.64\linewidth}
    \begin{flushleft}
        \subtitle{Postdoctoral Research Fellow}
        \affiliation{ \eoas{} \newline \ubc}
    \end{flushleft}
\end{minipage}
\hfill
\begin{minipage}[t]{0.36\linewidth}
    \begin{flushright}
        \textbf{email} \MAIL{\email}
        \\
        \textbf{website} \WEBSITE{\website}
        \\
        \textbf{ORCID} \ORCID{\orcid}
        \\
        \textbf{GitHub} \GITHUB{\github}
        \\
        \color{mygray}{Last modified: \today}
    \end{flushright}
\end{minipage}


\section{Professional Appointments}

\begin{cventries}
    \multiline{2022--on}{Postdoctoral Research Fellow}{%
        \eoas{} \newline \ubc{}
    }
    \multiline{2022}{Coding Trainer}{%
        Code to Communicate Initiative \newline
        NSF-funded bilingual program for grad-students
    }
    \multiline{2017--2019}{Assistant Professor of Practice}{%
        \fcefn,\newline \unsj, San Juan, Argentina
    }
    \multiline{2013--2015}{Student Teacher}{%
        \fceia,\newline \unr, Rosario, Argentina
    }
\end{cventries}

\section{Education}

\begin{cventries}
    \multiline{2017--2022}{PhD in Geophysics}{
        \fcefn,\newline \unsj, San Juan, Argentina
    }

    \multiline{2009--2015}{Licentiate in Physics}{
        \fceia,\newline \unr, Rosario, Argentina
    }
\end{cventries}


\section{Awards and Scholarships}

\begin{cventries}
    \singleline{2017--2022}{PhD Scholarship from CONICET}
    \singleline{2019}{%
        Early Career Scientist's Travel Support for EGU2020 General Assembly
    }
    \singleline{2016}{Fundación Josefina Prats Award: Licenciatura en Física}
    \singleline{2012}{%
        University Scholarship granted by Fundación del Banco de Santa Fe
    }
    \singleline{2012}{Santander Río National Award to Academic Merit 2012}
    \singleline{2012}{Josefina Prats Award: IFIR 2012}
\end{cventries}


\section{Certifications}

\begin{cventries}
    \singleline{2021}{Certified Software Carpentry Instructor}
\end{cventries}


\section{Teaching Experience}

\subsection{Undergraduate Courses}

\begin{cventries}
    \multiline{2022}{%
        EOSC 350: Environmental, Geotechnical, and Exploration Geophysics I
    }{%
        \eoas, \newline \ubc{}
    }
    \multiline{2017--2019}{Statistical Mechanics}{%
        Department of Geophysics and Astronomy,
        \newline \fcefn,\newline \unsj{}
    }
    \multiline{2017--2019}{Physics I}{%
        Department of Geophysics and Astronomy,
        \newline \fcefn,\newline \unsj{}
    }
    \multiline{2013--2015}{Physics I}{%
        Physics Department, \newline \ecen\newline \fceia, \newline \unr{}
    }
    \multiline{2013--2015}{Introduction to Science}{%
        Physics Department, \newline \ecen\newline \fceia, \newline \unr{}
    }
\end{cventries}

\newpage
\subsection{Workshops and Tutorials}

\begin{cventries}
    \multiline{2021}
        {Tutorial: Processing gravity and magnetic data with Harmonica}
        {
            \me, \andrea\ and \agustina{}
            \newline
            Transform21, Software Underground
            \newline
            \url{https://youtu.be/0bxZcCAr6bw}
        }
    \multiline{2020}
        {Introducción a Python para Científicxs}
        {
            \unsj, San Juan, Argentina
            \newline
            \url{https://youtu.be/LS_e9gqTM2s}
        }
    \multiline{2020}
        {Tutorial: From scattered data to gridded products using Verde}
        {
            \leo\ and \me{}
            \newline
            Transform 2020, Software Underground
            \newline
            \url{https://youtu.be/-xZdNdvzm3E}
        }
    \multiline{2017}
        {Taller Introductorio a LaTeX:\@Cómo producir documentos de calidad}
        {
            \agustina\ and \me{}
            \newline
            \unsj, San Juan, Argentina
        }
\end{cventries}

\section{Publications}

\subsection{Peer-reviewed scientific articles}

\begin{cventries}
    \paper{2021}
        {Gradient-boosted equivalent sources}
        {Geophysical Journal International}
        {\me\ and \leo}
        {10.1093/gji/ggab297}

    \paper{2020}
        {Pooch: A friend to fetch your data files}
        {Journal of Open-Source Software}
        {\leo, \me, \remirampin, \hugovankemenade, \matthewturk, \danshapero,
        \andersonbanihirwe\ and \johnleeman}
        {10.21105/joss.01943}

    \paper{2019}
        {Gravitational field calculation in spherical coordinates using
        variable densities in depth}
        {Geophysical Journal International}
        {\me, \agustina, \mario\ and \leo}
        {10.1093/gji/ggz277}

    \paper{2018}
        {Transient plate contraction between two simultaneous slab windows:
        Insights from Paleogene tectonics of the Patagonian Andes}
        {Journal of Geodynamics}
        {\guido, \agustina\ and \me}
        {10.1016/j.jog.2018.07.008}

    \paper{2017}
        {Analysis of the Illapel MW=8.3 thrust earthquake rupture zone using
        GOCE-derived gradients}
        {Pure and Applied Geophysics}
        {\orlando, \agustina, \mario, \folguera, \me\ and \wenjinchen}
        {10.1007/s00024-016-1376-y}

    \paper{2017}
        {Effective elastic thickness in the Central Andes. Correlation to
        orogenic deformation styles and lower crust high-gravity anomaly}
        {Journal of South American Earth Sciences}
        {\pichu, \guido, \mae, \marcos, \me, \paco\ and \fede}
        {10.1016/j.jsames.2017.11.021}
\end{cventries}


\section{Presentations}

\subsection{Invited speaker}

\begin{cventries}

\talk{2022}
    {Fatiando a Terra: Open-source tools for geophysics}
    {\gif, \eoas, \ubc}
    {Slides: \url{https://www.santisoler.com/2022-ubc-fatiando}}
    {}

\talk{2022}
    {Empowering science with open-source software}
    {Instituto de Astronomia, Geofísica e Ciências Atmosféricas, Universidade de São Paulo}
    {Slides: \url{https://www.santisoler.com/iag-usp-2022}}
    {}

\talk{2021}
    {Fatiando a Terra: Open-source tools for geophysics}
    {Geophysical Society of Houston}
    {
        \leo, \me\ and \agustina{}
        \newline
        Slides: \url{https://www.fatiando.org/2021-gsh}
    }
    {}

\end{cventries}

\subsection{Conference proceedings}

\begin{cventries}
\talk{2023}
    {Fatiando a Terra: Open-source tools for geophysics}
    {Canadian Exploration Geophysical Society (KEGS)}
    {
        \me\ and \lindsey{}

        Slides: \DOI{10.6084/m9.figshare.22151357}
    }
    {}

\talk{2021}
    {Gradient-boosted equivalent sources for gridding large gravity and
    magnetic datasets}
    {EGU21 General Assembly}
    {
        \me\ and \leo{}

        Talk: \DOI{10.6084/m9.figshare.14515188}
        \hspace{1em}
        Slide: \DOI{10.6084/m9.figshare.14461792}
    }
    {10.5194/egusphere-egu21-1276}

\talk{2020}
    {How to grid gravmag data with Harmonica}
    {Transform 2020, Software Underground}
    {Lightning talk: \url{https://youtu.be/NtBDf7d7mwM?t=4245}}
    {}

\talk{2020}
    {A better strategy for interpolating gravity and magnetic data}
    {EGU2020, General Assembly}
    {\me\ and \leo}
    {10.6084/m9.figshare.12217973}

\talk{2020}
    {Evaluating the accuracy of equivalent-source predictions using cross-validation}
    {EGU2020, General Assembly}
    {\me\ and \leo}
    {10.5194/egusphere-egu2020-15729}

\talk{2019}
    {Experiencias en el desarrollo de Fatiando a Terra}
    {Taller Argentino de Computación Científica}
    {}
    {10.6084/m9.figshare.10013006}

\talk{2019}
    {Gravitational fields of tesseroids with variable density}
    {LAPIS 2019}
    {\me, \agustina, \mario\ and \leo}
    {10.6084/m9.figshare.8242439}

\talk{2017}
    {Tesseroides con densidad variable: modelo directo con software libre}
    {I Congreso Binacional de Investigación Científica y V Encuentro de Jóvenes
    Investigadores}
    {\me, \agustina, \leo\ and \mario}
    {}

\talk{2017}
    {Magnetic characterization of the Loncopué trough, Argentina}
    {XX Congreso Geológico Argentino}
    {\agustina, \mario, \folguera, \guido\ and \me}
    {}

\talk{2017}
    {Magnetic characterization of the Loncopué trough, Argentina}
    {XX Congreso Geológico Argentino}
    {\agustina, \mario, \folguera, \guido\ and \me}
    {}

\talk{2017}
    {Anomalías gravimétricas y magnéticas corticales del sur de la provincia volcánica de La Payenia, asociadas al tearing de la Placa de Nazca y anomalías mantélicas}
    {XX Congreso Geológico Argentino}
    {Ana Astort, Bruno Colavitto, Lucía Sagripanti, \pichu, \me, \paco\ and \folguera}
    {}

\talk{2016}
    {Análisis flexural de la cuenca cretácico-paleógena del noroeste argentino. La subcuenca Lomas de Olmedo: zona de transición entre dos mecanismos de deformación distintos}
    {Primer Simposio de Tectónica Sudamericana}
    {\pichu, \me, \guido\ and \paco}
    {}

\talk{2016}
    {Crustal Magmatic Anomalies from the Southern Payenia Volcanic Plateau, Associated with The Nazca Plate Tearing and Plume Head from Gravimetric and Magnetic Data}
    {Primer Simposio de Tectónica Sudamericana}
    {Ana Astort, \paco, \pichu, \me, Andrés Echaurren and \folguera}
    {}

\end{cventries}


\section{Open-source Software Development}

\begin{cventries}
    \multiline{Since 2015}
        {Fatiando a Terra: Open source tools for geophysics}
        {Developer and maintainer \newline \url{https://www.fatiando.org}}
\end{cventries}

\section{Reviewer}

\begin{itemize}[label={}, leftmargin=*]
    \item \href{https://seg.org/Publications/Journals/Geophysics}{Geophysics}
    \item \href{https://openresearchsoftware.metajnl.com}{Journal of Open Research Software}
    \item \href{https://www.sciencedirect.com/journal/journal-of-applied-geophysics}{Journal of Applied Geophysics}
    \item \href{https://joss.theoj.org/}{Journal of Open-Source Software}
\end{itemize}

\section{Technical Skills}

\begin{description}
    \item[Programming] Python, Numba, Rust, bash, FORTRAN, C
    \item[Markup] Markdown, LaTeX, HTML, RST
    \item[WebDev] CSS, Bootstrap, Normalize, Static Site Generators (jekyll,
        urubu)
    \item[DevOps] GNU/Linux, Unix terminal, VIM, Neovim, git, GNU Make, SSH
    \item[Graphic Design] Inkscape, GIMP, imagemagick, Darktable
    \item[Other tools] Jupyter Notebooks, LibreOffice Suite, PGP, GitHub
        Actions, Nextcloud, maxima
\end{description}

\section{Languages}

\begin{description}
    \item[Spanish] Native
    \item[English] Advanced
    \item[Italian] Beginner
\end{description}

\end{document}
