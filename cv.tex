\documentclass[a4paper,12pt,sans,colorlinks]{moderncv/moderncv}
\usepackage[utf8]{inputenc}
\usepackage[scale=0.8]{geometry}
\recomputelengths
\usepackage{academicons}

% Configure moderncv
\moderncvstyle{classic}
\moderncvcolor{green}
\moderncvicons{awesome}

% Configure style of urls
\urlstyle{same}

% Redefine ORCID icon so it can be showed by ubuntu 18.04
\renewcommand*\orcidsocialsymbol{{\aiOrcid}~}

% Information
% -----------
\firstname{Santiago}
\familyname{Soler}
\email{santiago.r.soler@gmail.com}
\homepage{https://www.santisoler.com}
\social[github]{santisoler}
\social[orcid]{0000-0001-9202-5317}
\extrainfo{
    Last modified: \today{}
}
\title{
    \small
    CONICET, Argentina
    \\
    Instituto Geofísico Sismológico Volponi, UNSJ, Argentina
    \\
    Member of the
    \href{https://www.compgeolab.org}{Computer-Oriented Geoscience Lab}
}
% \address{street and number}{postcode city}
% \mobile{mobile}: optional
% \phone{phone}: optional
% \fax{fax}: optional
% \photo[64pt]{picture}
% \quote{Some quote}


% Authors
% -------
\newcommand{\me}{\href{https://www.santisoler.com}{Santiago Soler}}
\newcommand{\agustina}{\href{https://aguspesce.github.io}{Agustina Pesce}}
\newcommand{\andrea}{\href{https://www.andreabalza.com/}{Andrea Balza Morales}}
\newcommand{\guido}{Guido M. Gianni}
\newcommand{\leo}{\href{https://www.leouieda.com}{Leonardo Uieda}}
\newcommand{\mario}{Mario Giménez}
\newcommand{\remirampin}{Rémi Rampin}
\newcommand{\hugovankemenade}{Hugo vanKemenade}
\newcommand{\matthewturk}{Matthew Turk}
\newcommand{\danshapero}{Daniel Shapero}
\newcommand{\andersonbanihirwe}{Anderson Banihirwe}
\newcommand{\johnleeman}{John Leeman}
\newcommand{\orlando}{Orlando Álvarez}
\newcommand{\folguera}{Andrés Folguera}
\newcommand{\wenjinchen}{Wenjin Chen}
\newcommand{\pichu}{Héctor P.A. García}
\newcommand{\mae}{Marianela N. Lupari}
\newcommand{\marcos}{Marcos A. Sánchez}
\newcommand{\paco}{Francisco Ruiz}
\newcommand{\fede}{Federico G. Lince Klinger}

% doi function
\newcommand{\doi}[1]{
    \href{https://doi.org/#1}{#1}
}


\begin{document}

% Configure colors of links
\hypersetup{allcolors=[rgb]{0.34765625, 0.69921875, 0.296875}}

\maketitle

% Example of cventry
% \cventry[1em]
% {}  % date
% {} % degree/job title
% {} % institution
% {} % localization
% {} % grade (optional)
% {} % description (optional)

\section{Education}

\cventry[1em]
{Since 2017}
{PhD Student in Geophysics}
{
    Facultad de Ciencias Exactas, Físicas y Naturales,
    Universidad Nacional de San Juan
}
{San Juan, Argentina}
{}
{\emph{Advisor}: \mario{}, \emph{Co-advisor}: \leo{}}

\cventry[1em]
{2009--2015}
{Licentiate in Physics}
{
    Facultad de Ciencias Exactas, Ingeniería y Agrimensura,
    Universidad Nacional de Rosario
}
{Rosario, Santa Fe, Argentina}
{}
{
    \emph{Thesis}: Spectral methods for the determination of the Curie Depth
    Point and Elastic Thickness of Earth's Crust
    \\
    \emph{Advisor}: \mario{}
}


\section{Awards and Scholarships}

\cvline
{2017--2022}
{CONICET PhD Scholarship}

\cvline
{2019}
{Early Career Scientist's Travel Support for EGU2020 General Assembly}

\cvline
{2016}
{Fundación Josefina Prats Award - Licenciatura en Física}

\cvline
{2012}
{University Scholarship granted by Fundación del Banco de Santa Fe}

\cvline
{2012}
{Santander Río National Award to Academic Merit 2012}

\cvline
{2012}
{Josefina Prats Award - IFIR 2012}



\section{Teaching}

\cventry[1em]
{2021}  % date
{Certified Carpentries Instructor} % degree/job title
{The Carpentries} % institution
{} % localization
{} % grade (optional)
{} % description (optional)

\cventry[1em]
{2021}  % date
{Tutorial: Processing gravity and magnetic data with Harmonica}
{Transform21, Software Underground}
{} % localization
{} % grade (optional)
{\me{}, \andrea{} and \agustina{}} % description (optional)

\cventry[1em]
{2020}  % date
{Introducción a Python para Científicxs} % degree/job title
{Universidad Nacional de San Juan} % institution
{San Juan, Argentina} % localization
{} % grade (optional)
{\me{}} % description (optional)

\cventry[1em]
{2020}
{Tutorial: From scattered data to gridded products using Verde}
{Transform 2020, Software Underground}
{}
{}
{\leo{} and \me{}}

\cventry[1em]
{2017--2019}
{Physics I and Statistical Mechanics}
{
    Departamento de Geofísica y Astronomía,
    Facultad de Ciencias Exactas, Físicas y Naturales,
    Universidad Nacional de San Juan
}
{San Juan, Argentina}
{}
{Jefe de Trabajos Prácticos (assistant professor of practice)}

\cventry[1em]
{2013--2015}
{Introduction to Science and Physics I}
{
    Departamento de Física,
    Escuela de Ciencias Exactas y Naturales,
    Facultad de Ciencias Exactas, Ingeniería y Agrimensura,
    Universidad Nacional de Rosario
}
{Rosario, Santa Fe, Argentina}
{}
{Ayudante de Segunda (undergraduate assistant teacher)}


\section{Publications}

\subsection{Peer-reviewed scientific articles}

\cventry[1em]
{2021}
{Gradient-boosted equivalent sources}
{Geophysical Journal International}
{}
{}
{
    \me{} and \leo{}
    \\
    doi: \doi{10.1093/gji/ggab297}
}

\cventry[1em]
{2020}
{Pooch: A friend to fetch your data files}
{Journal of Open-Source Software}
{}
{}
{
    \leo{}, \me{}, \remirampin{}, \hugovankemenade{}, \matthewturk{},
    \danshapero{}, \andersonbanihirwe{} and \johnleeman{}
    \\
    doi: \doi{10.21105/joss.01943}
}

\cventry[1em]
{2019}
{
    Gravitational field calculation in spherical coordinates using variable
    densities in depth
}
{Geophysical Journal International}
{}
{}
{
    \me{}, \agustina{}, \mario{} and \leo{}
    \\
    doi: \doi{10.1093/gji/ggz277}
}

\cventry[1em]
{2018}
{
    Transient plate contraction between two simultaneous slab windows:
    Insights from Paleogene tectonics of the Patagonian Andes
}
{Journal of Geodynamics}
{}
{}
{
    \guido{}, \agustina{}, and \me{}
    \\
    doi: \doi{10.1016/j.jog.2018.07.008}
}

\cventry[1em]
{2017}
{
    Analysis of the Illapel MW=8.3 thrust earthquake rupture zone using
    GOCE-derived gradients
}
{Pure and Applied Geophysics}
{}
{}
{
    \orlando{}, \agustina{}, \mario{}, \folguera{}, \me{} and \wenjinchen{}
    \\
    doi: \doi{10.1007/s00024-016-1376-y}
}

\cventry[1em]
{2017}
{
    Effective elastic thickness in the Central Andes.
    Correlation to orogenic deformation styles and lower crust high-gravity
    anomaly
}
{Journal of South American Earth Sciences}
{}
{}
{
    \pichu{}, \guido{}, \mae{}, \marcos{}, \me{}, \paco{} and \fede{}
    \\
    doi: \doi{10.1016/j.jsames.2017.11.021}
}


\newpage
\subsection{Conference proceedings and presentations}

\cventry[1em]
{2021}
{Fatiando a Terra: Open-source tools for geophysics}
{Geophysical Society of Houston}
{}
{}
{\leo{}, \me{} and \agustina{}}

\cventry[1em]
{2021}
{
    Gradient-boosted equivalent sources for gridding large gravity and
    magnetic datasets
}
{EGU21 General Assembly}
{}
{}
{
    \me{} and \leo{}
    \\
    doi: \doi{10.5194/egusphere-egu21-1276}
}

\cventry[1em]
{2020}
{How to grid gravmag data with Harmonica}
{Transform 2020, Software Underground}
{}
{Lightning talk}
{}

\cventry[1em]
{2020}
{A better strategy for interpolating gravity and magnetic data}
{EGU2020, General Assembly}
{}
{}
{\me{} and \leo{}}

\cventry[1em]
{2020}
{
    Evaluating the accuracy of equivalent-source predictions using
    cross-validation
}
{EGU2020, General Assembly}
{}
{}
{\me{} and \leo{}}

\cventry[1em]
{2019}
{Experiencias en el desarrollo de Fatiando a Terra}
{Taller Argentino de Computación Científica}
{San Luis, Argentina}
{}
{doi: \doi{10.6084/m9.figshare.10013006}}

\cventry[1em]
{2019}
{Gravitational fields of tesseroids with variable density}
{LAPIS 2019}
{La Plata, Buenos Aires, Argentina}
{}
{
    \me{}, \agustina{}, \mario{} and \leo{}
    \\
    doi: \doi{10.6084/m9.figshare.8242439}
}

\cventry[1em]
{2017}
{Tesseroides con densidad variable: modelo directo con software libre}
{
    I Congreso Binacional de Investigación Científica y V Encuentro de Jóvenes
    Investigadores
}
{San Juan, Argentina}
{}
{\me{}, \agustina{}, \leo{} and \mario{}}

\cventry[1em]
{2017}
{Magnetic characterization of the Loncopué trough, Argentina}
{XX Congreso Geológico Argentino}
{}
{}
{\agustina{}, \mario{}, \folguera{}, \guido{} and \me{}}

\cventry[1em]
{2017}
{Magnetic characterization of the Loncopué trough, Argentina}
{XX Congreso Geológico Argentino}
{}
{}
{\agustina{}, \mario{}, \folguera{}, \guido{} and \me{}}

\cventry[1em]
{2017}
{
    Anomalías gravimétricas y magnéticas corticales del sur de la provincia
    volcánica de La Payenia, asociadas al tearing de la Placa de Nazca
    y anomalías mantélicas
}
{XX Congreso Geológico Argentino}
{}
{}
{
    Ana Astort, Bruno Colavitto, Lucía Sagripanti, \pichu{}, \me{}, \paco{}
    and \folguera{}
}

\cventry[1em]
{2016}
{
   Análisis flexural de la cuenca cretácico-paleógena del noroeste
   argentino. La subcuenca Lomas de Olmedo: zona de transición entre dos
   mecanismos de deformación distintos.
}
{Primer Simposio de Tectónica Sudamericana}
{}
{}
{\pichu{}, \me{}, \guido{}, and \paco{}}

\cventry[1em]
{2016}
{
    Crustal Magmatic Anomalies from the Southern Payenia Volcanic Plateau,
    Associated with The Nazca Plate Tearing and Plume Head from Gravimetric
    and Magnetic Data
}
{Primer Simposio de Tectónica Sudamericana}
{}
{}
{Ana Astort, \paco{}, \pichu{}, \me{}, Andrés Echaurren and \folguera{}}


\section{Open-source software}

\cventry
{Since 2015}
{Fatiando a Terra: Open source tools for geophysics}
{}
{}
{}
{
    Developer and maintainer
    \\
    \url{https://www.fatiando.org}
}

\section{Reviewer}

\cvlistitem{
    \href
    {https://openresearchsoftware.metajnl.com/}
    {Journal of Open Research Software}
}
\cvlistitem{
    \href
    {https://www.sciencedirect.com/journal/journal-of-applied-geophysics}
    {Journal of Applied Geophysics}
}
\cvlistitem{
    \href
    {https://joss.theoj.org/}
    {Journal of Open-Source Software}
}


\section{Technical Skills}

\cvline{Programming}
{Python, bash, FORTRAN, C}
\cvline{Markup}
{Markdown, LaTeX, HTML, CSS, Bootstrap}{
\cvline{Tools}
{
    GNU/Linux, Unix Terminal, git, GitHub, Jupyter Notebooks, GNU Make,
    Inkscape, GIMP, GMT, maxima
}

\section{Languages}

\cvlanguage{Spanish}{Native}{}
\cvlanguage{English}{Advanced}{}
\cvlanguage{Italian}{Beginner}{}

\end{document}
